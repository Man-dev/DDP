\chapter{Microgrid Reconfiguration in Action: Selected Applications} \label{ch3}

\section{Reconfiguration in Microgrid Combined with Unit Commitment }\label{ch3sec1}
This section will explain how use of reconfiguration in microgrids can be useful while solving a unit comittment problem in microgrid in order to further enhance it , and is mainly based on ``Microgrid operation and management using probabilistic reconfiguration and unit commitment" by ``Reza Jabbari-Sabet , Seyed-Masoud Moghaddas-Tafreshi, Seyed-Sattar Mirhoseini"\citep{Jabbari-Sabet2016328}. This paper was published in International Journal of Electrical Power \&  Energy Systems in February 2016. This paper is combining unit comittment and microgrid reconfiguration, or in other words, enhancing unit comittment problem in microgrids by allowing microgrid reconfiguration. They look at a day-ahead unit comittment problem from the perspective of microgrid manager, and try to find optimum unit commitment and optimum network topology (via reconfiguration) so as to maximize benefit or profit which the micogrid manager makes. Authors optimize this system for several cases, considering wind input to be probabilistic in nature, and then aggregate the results.\\
\subsubsection{Problem Formulation and Assumptions}
The benefit or profit can be written for the day ahead problem as
\begin{eqnarray}
Max: OF = \sum\limits_{t=1}^{24}(revenue(t)-cost(t))
\end{eqnarray}
The system selected in this paper is a 10 bus system, with 3 microturbines(MTs) (which run on conventional fossil fuels), 1 battery interfaced with an inverter, a wind generator, and this system is connected to grid through a single bus.(as shown in the Figure ~\ref{fig:pap1grid})
\begin{figure}[tbp]
  \centering
    \includegraphics[width=0.8\textwidth]{paper1grid.jpg}%paper1gridpaper1grid
    \caption[Paper 1 Microgrid System]{The microgrid model used for study in paper 1(image adopted from \cite{Jabbari-Sabet2016328})}
    \label{fig:pap1grid} 
\end{figure}

A total number of 8 loads are considered, with some of them being vital loads and others non vital. The authors try to perform this optimization by selecting their decision variables as all of the microturbine power outputs, the power exchanged with battery, the power exchanged with grid, and the network configuration (which essentially denotes which lines are open in the network, hence the network topology- several cases are given various numbers, and the variable $n\_topology$ stores that particular configuration number) for all the time periods, which results in a total of $24\times6=144$ decision variables, which includes 6 decision variables per hour (namely the power outputs of 3 MTs, the power exchange with battery, the power exchanged with grid, and the $n\_topology$ for that hour) for 24 hours.\\
In considering revenue and cost part of the objective function, the authors have included several sources of revenue and several components for cost: in the revenue part, the authors are considering two components: the revenue generated by feeding off the loads, and also the revenue generated through selling electricity to the grid (As can be seen in the equation ~\eqref{eq:pap1rev})
\begin{eqnarray}
\label{eq:pap1rev}	
revenue = R{load} + R{network}\\
R{load} = \sum\limits_{k=1}^{N_{load}}(\rho_{load,k}\cdot p_{load,k}\cdot t)\\
R_{network} = \rho_{sell-network}\cdot p_{sell-network}\cdotp t\\
\end{eqnarray}
where $\rho_{load,k}$ is the price of electricity for that time period for the $k^{th}$ in the microgrid, $p_{load,k}$ is the total load at the $k^{th}$ bus during the  interval, $t$ is equal to one hour (since the day ahead calculations are done hour by hour), $\rho_{sell-network}$ is the price at which microgrid can sell electric power to grid utility, and $ p_{sell-network}$ is the actual load the network sells to the grid during that hour. Note that all of these equations depict revenue generated in one period, and the revenue for each period can be calculated by using these equations. Note that both the prices are not fixed, so $\rho_{load,k}$ and $ \rho_{sell-network}$ vary according to the time of the day ($\rho_{buy-network}$ also varies according to the time of the day, but that will come in the costs). In addition, the vital and non vital loads buy electricity at different prices, with prices for vital loads being higher.\\
The costs include various costs: there are costs associated with power generation from all three microturbines as well as wind generator, the costs associated with buying power from the utility (grid), costs associated with power loss within the microgrid and the cost of switching (which comes into play because we might be reconfiguring the network, according to the optimization process). In equation form,
\begin{eqnarray}
\label{eq:pap1cost}
cost = \sum\limits_{j=1}^{N_{mt}}C_{mt}(j) + C_{wind} + C_{bat} + C_{network} + C_{loss} + C_{switching}
\end{eqnarray}
The above equation ~\eqref{eq:pap1cost} denotes the cost for one period. Total cost would be, then, the above expression summed over all of the intervals.
The models used for each of these costs are usually the standard ones: for microturbines, for instance, the cost is divided in terms of quadratic fuel cost, O\&M cost proportional to power output, startup cost (since this is a unit commitment problem), capital cost, and emission cost. While considering startup cost, it is considered that the startup cost increases as the microturbine is kept off for more and more time, the capital cost is given as annualized capital cost, and emission cost is considered to be proportional to the power generated.
The cost associated with wind power generation is calculated by calcutating two terms, O\&M term, which is taken constant (irrespective of the wind power generated), and capital cost, which is annualized. 
Battery cost also includes two terms, cost associated with the usage of battery, i.e. charging or discharging, which is proportional to the power battery exchanges with the microgrid for a period, which also includes a correction which accounts for battery degradation, i.e. cost increases as the battery gets older and has been used for more cycles, and second term which is O\&M cost, which is considered to be constant. 
Network cost is simply the amount of power bought from the utility for a particular interval times the price of the electricity. Note that the authors have not used a fixed price, but the price varies according to the time of the day. 
Switching cost is the result of the authors allowing reconfiguration of the system, and has two components, a capital cost, annualized, and another component which is proportional to the number of switching operations performed.\\
The constraints the authors have considered include standard constraints considered for unit comittment problem, i.e. the bus/node voltages should be in acceptable range, the currents should not exceed the line flow limits, the demand should be met inspite of losses (power balance), the generation limits for microturbine, the limits on battery power exchange, the limits on SOC of the battery, the limits on the grid power exchange etc. In addition, since this is a unit commitment problem, minimum uptime and downtime constraints are also added. Also, the topology constraint is added since this is a reconfiguration problem, which states that the network topology should always be radial (since this is a microgrid, akin to a distribution system).




\subsubsection{Methods}
When trying to solve this optimization problem, authors have considered the wind power output as well as load to be probabilistic and not deterministic, hence a probabilistic treatment to the problem is needed. The authors create multiple scenarios according to the probability distributions of wind speed and load, and solve a deterministic problem for each scenario. The results are then aggregated and analysed.\\
The wind turbine power depends on the wind speed and hence changes from scenario to scenario. For each period, 12 years' hourly wind speed data is available, and that is used to fit a Weibull distribution by relating the mean and standard deviation of the wind speed data to the parameters of Weibul distribution (equation~\eqref{eq:pap1wei}: here $r$ and $c$ are parameters of Weibull distribution dependent on the mean and variance of wind speed data).
\begin{equation}
\label{eq:pap1wei}
f(w) = \frac{r}{c}\left(\frac{w}{c}\right)^{r-1}\exp\left[-\left(\frac{w}{c}\right)^r\right]
\end{equation}
Then, for each scenario, a random number between 0 and 1 is generated for each hour, and on the Cumulative Distribution Function (CDF) of that hours' Weibul distribution, the value corresponding to CDF being that number is found and used as the wind speed for that particular hour. The wind power generated, then, is calculated by assuming quadratic dependence of power generated on the wind speed from cut in speed till the rated speed, and constant power output up to the cutoff speed after that. Loads are treated similarly, the only difference is that the probability distribution considered here is normal.\\
Once the wind power and load data is generated according to the scenario for each hour, it only remains to solve a deterministic unit comittment and reconfiguration problem, which is not necessarily an easy feat, due to presence of mixed integer programming required. The method employed in the paper by the authors is the particle swarm optimization (PSO). PSO technique is an evolutionary technique, which tries to search for a global optimum of a function. It starts by creating a set of random solutions within the allowed solution space, called particles. It then calculates the function value at all those solution points. Then it finds the particle best for each particle (this solution is the best function value it has reached so far- so for the first iteration, it will just be the function value) and the global best value- the best value any particle has reached so far (in the first iteration it will be the minimum value amongst the particles initiated). In the second phase it calculates a 'velocity' for each particle, which is calculated according to the difference between the value of the function at current position of a particle, and the particle best, and also the difference between the value of the function and the global best. Both these are multiplied by a random number between zero and one, and a weight given to each of these differences (usually equal weight is given- 2 for each term), and then these terms are added to current velocity in order to update it(equation ~\eqref{eq:pap1pso1}). Also the particle position is calulated according to the velocity(equation ~\eqref{eq:pap1pso2}), and hence each particle reaches at a new solution.
\begin{eqnarray}
\label{eq:pap1pso1}
v_{new} = v_{old} + c_1*r_1*(particle\_best - current) + c_2*r_2*(global\_best - current)
\end{eqnarray}
\begin{eqnarray}
\label{eq:pap1pso2}
position = position + velocity
\end{eqnarray}
In each new iteration, if applicable, particle best and global best value are updated, and those solutions are stored. Finally, when either maximum number of iterations are reached, or the particles reach a best solution and hence stop their movement, the method ends, and the gbest value and the corresponding solution are given as the output.
Generally, PSO is designed to work when we have an unconstrained optimization problem. Here though, there are many constraints which need to be satisfied, and hence if solution reached is violating any of the constraints (after calculating the losses  by doing forward/backward sweep load flow analysis), a penalty function is added to the objective function.\\
After calculating the optimum solution for the UC and reconfiguration problem for each scenario, such scenarios have to be aggregated to get an overall picture. This scenario aggregation is done using expected value. Also, coefficient of variance (which is standard deviation, relative to the mean) is computed, and if it is small, then the results are consistent. Since the topology code ($n_{topology}$) is a discrete variable, it is aggregated not by using expected value, but by taking the value which is seen the most i.e. taking the mode.
\subsubsection{Results and Discussion}
The authors find that there are two configurations of the network which are most commonly seen. It is seen that the MG sells electricity to the grid for most of the time, but buys near it's peak. The benefit by using this probabilistic reconfiguration and UC turns out to be 49,891 cents, with 3270 cents standard deviation. If only UC is used, 53,068 cents mean value for benefit is achieved, with standard deviation 13,863 cents. While it seems that the benefit achieved is less due to reconfiguration plus UC, where this method seems to be gaining is accuracy. The authors have done the calculations also on actual load and wind data for that particular date in 2012, and in both cases the benefit achieved turns out to be around 50,000 (50,691 in UC and reconfiguration, and 50,266 in case of only UC), so reconfiguration plus UC value is much more closer to reality and hence better prediction of real life situation. The authors have also varied the number of scenarios, and it turned out that at lower scenarios the coefficient of variation is higher, while it becomes low and converges when then number of scenarios get high, and is less than 1 percent when the number of scenarios is higher than 40.



\section{Electric Vehicles and Microgrid Reconfiguration}\label{ch3sec2}
This section explains how electric vehicles (EV) and reconfiguration can be integrated in order to benefit the microgrid by reducing the costs incurred, and is based on ``Efficient integration of plug-in electric vehicles via reconfigurable microgrids" by ``Abdollah Kavousi-Fard, Amin Khodaei"\citep{Kavousi-Fard2016653}. This paper was published in Energy in September 2016. This paper deals with microgrid reconfiguration from economic perspective, similar to previous paper, but considers presence of plug-in electric vehicles in addition to reconfiguration. The authors try to minimize the operating costs of running the microgrid and analyses the effects of reconfiguration and plug in electric vehicles on the overall cost of the microgrid. The analysis is done stochastically, by generating various scenarios and then solving the problem deterministically for each scenario. The authors have also considered the presence of various renewable generators while performing this analysis.
\subsubsection{Problem Formulation and Assumptions}
The system considered in this paper is a modified 32-bus IEEE system, with 1 PV source, 2 wind sources, 2 microturbines, 1 fuel cell, and two vehicular fleets(see figure ~\ref{fig:pap2grid}).
\begin{figure}[tbp]
  \centering
    \includegraphics[width=0.85\textwidth]{paper2grid.jpg}%paper1gridpaper1grid
    \caption[Paper 2 Microgrid System]{The microgrid model used for study in paper 2(image adopted from \cite{Kavousi-Fard2016653})}
    \label{fig:pap2grid} 
\end{figure}
The authors cast the optimization of minimizing operating cost. The operating cost is divided in several components, and is given by equation ~\eqref{eq:pap2cost}
\begin{eqnarray}
\label{eq:pap2cost}
Cost = C^{SW} + C^{DER} + C^{PEV} + C^{Rel} + C^{Sub}
\end{eqnarray}
Here $C^{SW}$ denotes the cost due to switching, and is proportional to the number of switching operations performed. It is important to note that the authors have limited the number of allowed switching operations per day, which comes from the reliability of switches point of view, stating that the switches are guaranteed to function normally up to a certain number of switching instances. $ C^{DER}$ denotes the operational cost incurred by operating the distributed generators in the microgrid, which includes both conventional and non-conventional sources. It is considered to be directly proportional to the power generated by the respective DER (as opposed to say using quadratic cost curve for microturbine). $C^{PEV}$ is the operational cost as a result of including and operating PEVs in the microgrid. This term is comprised of the charging/discharging costs associated with the connected PEVs, and the costs associated with V2G (vehicle to grid technology, essentially using the electric vehicle as a grid-connected baterry) as well as the due to the degradation this causes of the batteries, which is measured in terms of W\"{o}hler curve, which relates the number of charging-discharging cycles and the depth of discharge of a battery, and qualitatively can be described as the curve which gives us the number of cycles before the battery is expected to fail, and its DoD (depth of discharge). $C^{Sub}$ is the price associated with buying electricity from the connected grid. $C^{Rel}$ is a measure of reliability of a system and in the case of this paper, is expected customer interruption cost, and  denotes the economic loss faced if the load is unmet, by calculating the expected load for that profile. \\
The decision variables here are the configuration of network i.e. the statuses of sectionalizing and tie switches, the power outputs of various distributed energy resources, power exchanged with PEV fleet, the schedule of PEV fleet and the number of vehicles in it, and power exchanged with the grid: all of these for each time period (each hour). Hence it is a mixed integer problem.\\
There are several constraints which are considered while solving this problem. There are some commonly used constraints, like the limits on generation from the distributed generators (especially the ones based on fuel), the power flow equations, constraints stating that the voltage at any bus should not be too far away from 1 p.u., thermal limits on feeders (line flow limits) - these are generally used in almost every power system problem. Then there are some specific constraints also, which may or may not be present/considered in most problems. One constraint which is considered on the mains supply is that the mains supply is limited (there exists an upper limit on the power provided to a load). Then the authors have also considered the number of switching operations in a day to be limited. This constraint comes from the expected life of the switches, and the interruptible operations, after reserving some switching actions for things like fault detection. Then there are certain constraints associated with the PEV fleet. These include the condition that the vehicles in PEV fleet, if connected, can only be in charging, discharging, or idle mode, the limits on charging and discharging rate achievable by the fleet, the energy balance for the fleet of PEVs etc. In addition, there are constraints on the state of charge (SOC) of the batteries in PEVs. One of the constraints is that the SoC in the beginning of the day should be equal to the one at the last period, and that the PEV should be fully charged at the beginning of the first time it leaves the charging station during the day. Since this is a microgrid, radial structure is ensured (this is enforced as a constraint), even after the reconfigurations.\\
There are many uncertainties considered in the problem formulation, and these are taken care of by scenario generation and then analysis of each scenario, for example, the active and reactive loads at all the buses, the power output of wind and photovoltaic sources, the time of departure of PEV fleet (after which, the PEVs won't be available for some time), and also the market price of energy.\\
\subsubsection{Methods}
There are four main steps used to reach the optimal solution and analyse the problem. Firstly, many scenarios are generated randomly, which account for the inherent uncertainty present in many of the system parameters. Then these scenarios are aggregated and the number of scenarios to be considered is reduced to lessen the computational load. Then actual optimization is performed, and finally, the results are again aggregated in order to get a single value.\\
The scenarios used to analyse the system are generated randomly, using the Roulette Wheel Mechanism (RWM). In this method, each parameter where uncertainty exists is represented by a probability distribution function (PDF) and that PDF is then divided in several levels, each level corresponding to a different amount of error introduced and different probability level, which separates that particular value from the forecasted value (see figure \ref{fig:pap2rwm}).
\begin{figure}[tbp]
  \centering
    \includegraphics[width=0.7\textwidth]{paper2rwm.jpg}%paper1gridpaper1grid
    \caption[Roulette Wheel Mechanism]{Probability distribution function (PDF) is divided into multiple levels in Roulette Wheel Mechanism (image adopted from \cite{Kavousi-Fard2016653}).}
    \label{fig:pap2rwm} 
\end{figure}
Now, a random number is selected between 0 and 1, and then according to which probability level it falls into, an error is introduced in the forecasted value. Such process is repeated for each variable to generate a single scenario. Such many scenarios are generated (in the paper, 1000 scenarios are generated initially.)\\
Analysing a very high number of scenarios generally is computationally heavy, and there is not much point in analysing closely spaced scenarios, giving only marginal differences in results/ giving same results: hence the authors then reduce the number of scenarios by aggregating the scenarios, essentially finding similar and closely spaced scenarios, or scenarios which have a very low probability and reject those to reduce the total number of scenarios. The way these authors mathematically select the scenarios to reject is that they first compute the distance between all the pairs of scenarios (similar to Eucleadian distance -  for each parameter, adding square of the difference between the values, and then finally taking the square root: see equation \eqref{eq:pap2dee'}), and then they compute the value of the probability multiplied by the least distance from any other scenario for every scenario. The scenario having the least such value is rejected (see equation \eqref{eq:pap2rej}), and accordingly probability of the closest state is modified. Such process is repeated until the required number of scenarios is reached. The 1000 scenarios initially taken by the authors are reduced to 20 by this process.\\
\begin{eqnarray}
\label{eq:pap2dee'}
D_{ss'} = \sqrt{\sum\limits_{g}^{w}(r_{sg}-r_{sg'})^2}\\
\label{eq:pap2rej}
s_{reject} = \min(P_s \times D_{sl}),\\
D_{sl} = \min{D_{ss'}}
\end{eqnarray}
Here $s$ and $s'$ are scenarios, $w$ is the number of variables generated randomly for each scenario, $r_{sg}$ is $g^{\text{th}}$ variable randomly generated for scenario $s$, $P_s$ is the probability of occurrence of $s$, and $D_{sl}$ is the least distance $s$ has from any other scenario.\\ 
After the reduced scenarios are generated, each scenario is deterministically solved, and in this paper, using SAMCSA(Self-Adaptive Clonal Selection Algorithm), which is an evolutionary algorithm based on artificial immune system. It is modeled based on the response of immune system to pathogens. Initially a set of random solutions are generated (called population), then they are ranked according to their `fit' or `affinity', which is more if cost (objective function value) is less, and then top solutions are selected and cloned (with more clones created for solutions with higher affinity), and then these clones are modified a little bit, in a process called `hypermutation'. Again affinity of all these clones is computed, and then top affinity solutions are picked to be part of population for the next iteration, and the least affinity solutions are replaced by new random solutions. Such iterations are repeated several times. This is the basic clonal selection algorithm. The self-adaptive clonal selection algorithm used in this paper works similarly, and while hypermutating, certain techniques are used to improve the performance of the technique.\\
After optimization is done for each scenario, the answers are finally aggregated, and a single solution is obtained. This aggregation is done taking into account the results of each scenario and the probability of that scenario, i.e. the expected value of the optimum solution is found out and is designated as the final solution.
\subsubsection{Results and Discussion}
The solution is computed considering different cases, i.e. solving just as a dispatch problem, allowing the effect of PEVs and considering reconfiguration etc, so that the effects of these will be clearly seen on the solution (cost) achieved. Compared to the initial condition, where no optimization has been performed, and which has a cost of 50,111.9 euros, the scenario with reconfiguration plus PEV yields the cost to be minimum amongst the cases studied, 48,969.5 euros. This cost is for the whole day. The authors have also computed the power losses observed in the system, and found that the losses are also minimum in the last case where PEVs as well as reconfiguration is considered. The voltage at all the buses is also observed to stay within 0.1 deviation. The interesting thing to observe in this paper is that there is a huge drop in cost as well as power loss when the DGs are allowed to be turned off for some period. The losses increase slightly when PEVs are allowed (not too surprising, since new net load is added), but this is compensated when reconfiguration of network is allowed. Also interesting to note is that even though losses are increasing when we add PEVs in the system, the cost associated is always decreasing when we allow for PEV or reconfiguration.



\section{Reconfiguration for Feeding Maximum Loads During Faults}\label{ch3sec3}
This section explains how reconfiguration can be useful in microgrids when there is an abnormal condition or system failure, and is mainly based on ``Intelligent control algorithms for optimal reconfiguration of microgrid distribution system" by ``Farshid Shariatzadeh, Nikhil Kumar, and Anurag K Srivastava"\citep{Shariatzadeh2015}. This paper was published in  Industry Applications Society Annual Meeting, 2015 IEEE in December 2015. This paper is a bit different from the previous two, and doesn't take economic consideration as the primary objective to be met/satisfied. Rather, the authors are considering shipboard power system here, and in case of fault, they are trying to minimize the loss caused. That is, they try to feed to maximum number of loads/high priority loads by reconfiguring the shipboard power system. The authors have also implemented this in real time using real time digital simulator.
\subsubsection{Problem Formulation and Assumptions}
The system used in this paper for analysis is a 13 bus shipboard system model (see figure \ref{fig:pap3grid}).
\begin{figure}[tbp]
  \centering
    \includegraphics[width=0.75\textwidth]{paper3grid.jpg}%paper1gridpaper1grid
    \caption[Paper 3 Microgrid System]{The microgrid model (shipboard system) used for study in paper 3(image adopted from \cite{Shariatzadeh2015})}
    \label{fig:pap3grid} 
\end{figure}
There are total 8 generators (out of which 3 are distributed generators with smaller power capacities compared to the others) and 13 loads (out of which 6 are non-vital, 5 are semi-vital and 2 are vital. All of the loads and generators come with circuit breakers, and in addition, all of the lines also have tie breakers, and all of these enable reconfiguration in the system.\\
This shipboard system should provide power to maximum loads even in case of fault, and the authors have analysed the system with respect to multiple objective functions. The first is simply the amount of load met, where the total amount of load met is maximized (equation \eqref{eq:pap3of1}). Second objective takes into account that some of the loads are vital, whereas other are semi-vital or non-vital (equation \eqref{eq:pap3of2}). In this case, the authors assign a weight to each load according to their vitality (such that vital loads always have the load multiplied by weight value higher than semi vital loads, which in turn have higher value than non vital loads). The objective, in this case, is to maximize the sum of loads met, multiplied by their weights. In the third objective checked, both of these are combined, assigning weighing factors to both of these (equation \eqref{eq:pap3of3}). \\
\begin{eqnarray}
\label{eq:pap3of1}
F_1 = \max\sum_i x_iL_i
\end{eqnarray}
\begin{eqnarray}
\label{eq:pap3of2}
F_2 = \max\sum_i P_ix_iL_i
\end{eqnarray}
\begin{eqnarray}
\label{eq:pap3of3}
F_3 = \max W_M(\sum_i x_iL_i)+W_L(\sum_i P_ix_iL_i)
\end{eqnarray}
Here $L_i$ is the load at $i^{\text{th}}$ bus, $x_i$ corresponds to the status of circuit breaker which connects $L_i$ to the microgrid (shipboard power system), $P_i$ is the weight given to the load according to the vitality of the load, and $W_M$ and $W_L$ are the weightages given to both the terms.\\
The decision variables are chosen as the statuses of all the circuit breakers. The constraint chosen is that the cumulative power generated should be  more than the load fed, in all three objective functions. The authors perform this optimization in several cases, where different faulty buses are considered in each case.
\subsubsection{Methods}
The authors solve the optimization problem by two methods: one of them is genetic algorithm and other one is particle swarm optimization. In genetic algorithm, a set of initial solutions are randomly generated, called chromosomes. The individual elements of a solution are called genes. Each chromosome is now checked for their fitness value and ranked accordingly. Top chromosomes are picked up, and these generate child chromosomes. While doing this, either mutation (altering of one or more genes from the chromosome), or crossover (where a child has genes from more than one parent, i.e. a point is selected on two chromosomes, and the data beyond that point is swapped to generate two children from these two parents) might occur -  this process is called evolution. New generation thus generated is taken as the population for the next iteration. This process is repeated till the termination criteria is reached. Second algorithm used in this paper is particle swarm optimization (similar to the one in first paper, described in section \ref{ch3sec1}). Since in this case the decision variables can only take the values 0 and 1, the particle velocity in a dimension represents the probability of that particle having state 1 in that dimension. To decide the value of a particular component of particle position, it is compared with the sigmoid function value (see equation \eqref{eq:pap3sig}) the velocity of the particle in that dimension. If the sigmoid function value is greater than the random number, then the position is set to 1, else it is set to zero.\\
\begin{equation}
\label{eq:pap3sig}
S(x) = \frac{1}{1+e^x}
\end{equation}
Overall, for each fault, the breaker statuses are updated (trip signals are sent and accordingly relevant circuit breakers trip and open), and if no zone (which means bus combined with connected lines) has net negative power balance (load is greater than generation - the power balance can be found out, conducting a power flow) then the system is reconfigured using PSO or GA, and then the relevant breakers are opened or closed.
\subsubsection{Results}
According to the objective chosen, the partial load is shed to meet the power balance requirements, i.e. if the maximum load objective is selected, in some cases a vital load is shed and more MW is supplied in total, but when maximum priority objective is applied to the same system, a semi vital load is shed and vital load is fed, even though the net MW fed to all the loads come out to be lesser. In some other cases though, the results are identical, and in fact in some cases even in high priority objective a vital load is shed, because there is simply not enough generation capacity to feed that load. It is also worthwhile to note that the authors have got the same optimum configurations by using either GA or PSO, but the PSO has the lower computation time.\\
\subsubsection{Real Time Implementation}
The shipboard power system used in this study is also implemented in real time, using real time digital simulator(RTDS). For controlling the status of circuit breakers, a dSPACE controller is used, which can be programmed to perform the reconfiguration by interfacing it with a computer and programming it to do so in MATLAB. Shipboard power system is modelled in RSCAD, and when a fault occurs, it sends signals to the dSPACE controller, and then if there is a zone with negative power balance then the controller performs reconfiguration (optimization) and sends back the new statuses of various circuit breakers to the RTDS.




%%% Local Variables: 
%%% mode: latex
%%% TeX-master: "../mainrep"
%%% End: 
