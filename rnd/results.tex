\chapter{Future Work} \label{ch4}

%General trends observed in literature review
Reconfiguration in microgrids generally enhances the operation of microgrid by adding a bunch of new decision variables at the disposal of microgrid operator. Reconfiguration in microgrids has been shown to improve various objectives relating to microgrid. 
%State shortcomings of the research/ future work given in papers/challenges
There are certain challenges that still remain, however. The effect of reconfiguration on the transients in the network needs to be studied - usually the switching in electrical grid corresponds to high frequency oscillations. For switching operations like change in load or generation, the grid is well-suited to cope up, but for microgrids, some study needs to be performed\citep{mgrj18}. Another big challenge is the investment needed to set up the reconfigurable microgrid. Reconfiguration which considers the whole system requires a central control system, so costs of the sectionalizing and tie switches, as well as the communication system and central control system needs to be accounted for. While this system will probably be used for other purposes also, its impact on the investment required cannot be ignored, and it should be studied if reconfiguration in microgrids is really beneficial, and in what cases it is so. Some options like central processing but manual switching or local reconfiguration not requiring a central controller can be considered\citep{mgrj40}. As far as the economics are concerned, the impact of reconfiguration on the wear and tear of switches also needs to be taken into account\citep{mgrj40}. Impacts of reconfiguration on microgrid protection schemes and fault current levels can also be investigated. Reconfiguration may also lead to certain new problems like congestion caused due to uncertainty in DER production and mistaken network configuration due to faulty forecasts. Also, gathering and processing data using a network monitoring scheme will also require a few modifications if reconfiguration is taken into account.\\
%conclusion - 'these would be the areas on which to concentrate upon'
Going forward, there are several topics to investigate. There is not much work done regarding how protection systems in microgrid should be set up / how they work, so protection of reconfigurable microgrid could be one potential area which could be further studied. Another such area is optimal reconfiguration - microgrid reconfiguration as an optimization has been studied for various objectives like power loss minimization, operation cost minimization, vulnerability minimization, improving voltage stability etc., and an overview of this work can be seen in chapter ~\ref{ch2}. However, this topic needs to be further studied and can be looked into, especially from the perspective of multiobjective ooptimization, and including the uncertainty due to various factors such as DERs and loads.
%%% Local Variables: 
%%% mode: latex
%%% TeX-master: "../mainrep"
%%% End: 
